\documentclass{beamer}
\mode<presentation>
\usetheme{CambridgeUS}
\usepackage[russian]{babel}
\usepackage[utf8]{inputenc}
\usepackage[T2A]{fontenc}
\usepackage{sansmathaccent}

\usepackage{verbatim}
\usepackage{alltt}

\pdfmapfile{+sansmathaccent.map}
\title[Подпрограммы]{Функции и процедуры}
\author{Наумов Д.А., доц. каф. КТ, ИТГД }
\date[01.04.2019] {Алгоритмические языки и программирование, 2019}

\begin{document}

%ТИТУЛЬНЫЙ СЛАЙД
\begin{frame}
  \titlepage
\end{frame}
  
%СОДЕРЖАНИЕ ЛЕКЦИИ
\begin{frame}
  \frametitle{Содержание лекции}
  \tableofcontents  
\end{frame}
  
%РАЗДЕЛ 1
\section{Подпрограммы: процедуры и функции}
\subsection{Подпрограммы}
\begin{frame}
\begin{block}{Подпрограммы}
идентифицированная часть компьютерной программы, содержащая описание определённого набора действий, которая может быть многократно вызвана из разных частей программы.
\end{block}
Назначение подпрограмм:
\begin{itemize}
\item выделить целостную подзадачу, имеющую типовое решение;
\item сделать программу более понятной и обозримой.
\end{itemize}
Преимущества подпрограмм:
\begin{enumerate}
\item декомпозиция сложной задачи;
\item уменьшение дублирования кода;
\item возможность повторного использования кода;
\item разделение задач между исполнителями или стадиями проекта;
\item сокрытие деталей реализации;
\item упрощение отладки.
\end{enumerate}
\end{frame} 

\subsection{Описание функций}
\begin{frame}[fragile]
Виды подпрограмм в языке Pascal: \textbf{функции} и \textbf{процедуры}. Синтаксическая форма описания функции:
\begin{alltt}
1 function <ИмяФункции>([<СписокФормПарам>]):<ТипВозврЗнач>;
2     [<РазделОписаний>]
3 begin
4   <Оператор1>;
5   <Оператор2>;
6   ...;
7   <Оператор3>;
8 end;
\end{alltt}
\begin{itemize}
\item $<$ ИмяФункции $>$ - имя функции, идентификатор;
\item $<$ СписокФормПарам $>$ - список формальных параметров с указанием их типа.
\item $<$ ТипВозврЗнач $>$ - тип значения, возвращаемого функцией.
\item $<$ РазделОписаний $>$ - раздел описаний локальных меток, констант, переменных, типов данных, процедур и функций.
\end{itemize}
\end{frame}

\begin{frame}[fragile]
Пример описания функции для вычисления степени с натуральным показателем:
\begin{alltt}
1 function CalcPower(x: real; n: integer):real;
2 var 
3   i: integer;
4   p: real;
5 begin
6   p := 1;
7   for i := 1 to n do 
8       p := p * x;
9   CalcPower := p; (*возвращаем значение*)
10 end;
\end{alltt}
\end{frame}

\begin{frame}[fragile]
При обращении к функции происходит:
\begin{itemize}
\item вычисление значений фактических параметров (слева направо);
\item подстановка значений фактических параметров на место формальных параметров;
\item выполнение операторов тела функции;
\item возврат значения функции в основную программу;
\item возврат управления в точку вызова;
\end{itemize}
Вычисление выражения $z = x^{5} + (x+1)^{3}$:
\begin{alltt}
11 var x, z: real; 
12 begin
13   x := 1.001;
14   z := CalcPower(x, 5) + CalcPower(x+1, 3);
15   writeln('x=', x:6:4, 'z=', z:6:4);
16 end.
\end{alltt}
\begin{itemize}
\item вызов функции должен осуществляться в некотором выражении (иначе "потеряется" возвращаемое значение);
\item в теле функции должен выполниться хотя бы раз оператор вида \textit{ИмяФункции := Значение} (иначе возвращаемое значение будет неопределено).
\end{itemize}
\end{frame}

\subsection{Описание процедуры}
\begin{frame}[fragile]
Синтаксическая форма описания процедуры:
\begin{alltt}
1 procedure <ИмяПроцедуры>([<СписокФормПарам>]);
2     [<РазделОписаний>]
3 begin
4   <Оператор1>;
5   <Оператор2>;
6   ...;
7   <Оператор3>;
8 end;
\end{alltt}
\begin{itemize}
\item $<$ ИмяПроцедуры $>$ - имя функции, идентификатор;
\item $<$ СписокФормПарам $>$ - список формальных параметров с указанием их типа.
\item $<$ РазделОписаний $>$ - раздел описаний локальных меток, констант, переменных, типов данных, процедур и функций.
\end{itemize}
\end{frame}

\begin{frame}
Виды формальных параметров:
\begin{center}
\begin{tabular}{p{0.15\linewidth}||p{0.1\linewidth}|p{0.15\linewidth}|p{0.15\linewidth}|p{0.15\linewidth}}
Вид & Ключевое слово & Может изменяться в п/п & Фактический параметр изменится & Передается\\
\hline параметр-значение & - & да & нет & значение\\
\hline параметр-переменная & var & да & да & адрес\\
\hline параметр-константа & const & нет & нет & адрес\\
\end{tabular}
\end{center}
\end{frame}

\begin{frame}[fragile]
Процедура сортировки массива:
\begin{alltt}
1 type
2   TIndex = 1..10;
3   TElem = real;
4   TArray = array[TIndex] of TElem;
5 procedure Sort(var V:TArray; const n:TIndex); 
6 var
7   i, j: TIndex;
8   tmp:  TElem;
9 begin
10  for i := 1 to n-1 do
11    for j := i + 1 to n do
12      if V[i] > V[j] then
13      begin
14        tmp := V[i]; V[i] := V[j]; V[j] := tmp;
15      end;
16 end;
\end{alltt}
\end{frame}

\begin{frame}[fragile]
При обращении к процедуре происходит:
\begin{itemize}
\item вычисление значений фактических параметров (слева направо);
\item подстановка значений фактических параметров на место формальных параметров;
\item выполнение операторов тела процедуры;
\item возврат управления в точку вызова;
\end{itemize}
Сортировка массива:
\begin{alltt}
11 var vector: TVector; 
12 var i: TIndex; 
13 begin
14   for i := 1 to 10 do
15      vector[i] := random(100);
16   Sort(vector, 10);
17 end.
\end{alltt}
Вызов процедуры должен осуществляться в отдельном операторе.
\end{frame}

%РАЗДЕЛ 2
\section{Варианты заданий}
\begin{frame}{Варианты заданий}
\textbf{Вариант 1}. Дана целочисленная прямоугольная матрица. Определить:
\begin{enumerate}
\item количество строк, не содержащих ни одного нулевого элемента (оформить в виде функции);
\item максимальное из чисел, встречающихся в заданной матрице более одного раза (оформить в виде процедуры).
\end{enumerate}
\textbf{Вариант 2}. Дана целочисленная прямоугольная матрица. Определить:
\begin{enumerate}
\item количество столбцов, не содержащих ни одного нулевого элемента (оформить в виде функции); 
\item Характеристикой строки целочисленной матрицы назовем сумму ее положительных четных элементов. Переставляя строки заданной матрицы, расположить их в соответствии с ростом характеристик (оформить в виде процедуры).
\end{enumerate}
\end{frame} 

\begin{frame}{Варианты заданий}
\textbf{Вариант 3}. Дана целочисленная прямоугольная матрица. Определить:
\begin{enumerate}
\item количество столбцов, содержащих хотя бы один нулевой элемент (оформить в виде функции);
\item номер строки, в которой находится самая длинная серия одинаковых элементов (оформить в виде процедуры).
\end{enumerate}
\textbf{Вариант 4}. Дана целочисленная квадратная матрица. Определить:
\begin{enumerate}
\item произведение элементов в тех строках, которые не содержат отрицательных элементов (оформить в виде функции); 
\item максимум среди сумм элементов диагоналей, параллельных главной диагонали матрицы (оформить в виде процедуры).
\end{enumerate}
\end{frame} 

\begin{frame}{Варианты заданий}
\textbf{Вариант 5}. Дана целочисленная квадратная матрица. Определить:
\begin{enumerate}
\item сумму элементов в тех столбцах, которые не содержат отрицательных элементов (оформить в виде функции);
\item минимум среди сумм модулей элементов диагоналей, параллельных побочной диагонали матрицы (оформить в виде процедуры).
\end{enumerate}
\textbf{Вариант 6}. Дана целочисленная квадратная матрица. Определить:
\begin{enumerate}
\item такие k, что k-я строка матрицы совпадает с k-м столбцом (оформить в виде процедуры); 
\item найти сумму элементов в тех строках, которые содержат хотя бы один отрицательный элемент (оформить в виде функции).
\end{enumerate}
\end{frame}

\begin{frame}{Варианты заданий}
\textbf{Вариант 7}. Дана целочисленная квадратная матрица. Определить:
\begin{enumerate}
\item сумму элементов в тех столбцах, которые не содержат отрицательных элементов (оформить в виде функции);
\item минимум среди сумм модулей элементов диагоналей, параллельных побочной диагонали матрицы (оформить в виде процедуры).
\end{enumerate}
\textbf{Вариант 8}. Дана целочисленная прямоугольная матрица. Характеристикой столбца целочисленной матрицы назовем сумму модулей его отрицательных нечетных элементов. 
\begin{enumerate}
\item переставляя столбцы заданной матрицы, расположить их в соответствии с ростом характеристик (оформить в виде
процедуры); 
\item Найти сумму элементов в тех столбцах, которые содержат хотя бы один отрицательный элемент (оформить в виде функции).
\end{enumerate}
\end{frame}

\begin{frame}{Варианты заданий}
\textbf{Вариант 9}. Дана целочисленная квадратная матрица. Определить:
\begin{enumerate}
\item сумму элементов в тех столбцах, которые не содержат отрицательных элементов (оформить в виде функции);
\item минимум среди сумм модулей элементов диагоналей, параллельных побочной диагонали матрицы (оформить в виде процедуры).
\end{enumerate}
\textbf{Вариант 10}. 
\begin{enumerate}
\item Коэффициенты системы линейных уравнений заданы в виде прямоугольной матрицы. С помощью допустимых преобразований привести систему к треугольному виду (оформить в виде процедуры); 
\item Найти количество строк, среднее арифметическое элементов которых меньше заданной величины (оформить в виде функции).
\end{enumerate}
\end{frame}

\end{document}
