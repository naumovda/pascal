\documentclass[oneside, final, 14pt]{extreport}
\usepackage[utf8]{inputenc}
\usepackage[russian]{babel}
\usepackage{vmargin}
\setpapersize{A4}
\setmarginsrb{2cm}{1cm}{1cm}{1cm}{0pt}{0mm}{0pt}{13mm}
\usepackage{indentfirst}
\usepackage{amsmath}
\usepackage{graphicx}

\newcounter{ticket}
\setcounter{ticket}{1}

\begin{document}

% Билет 1
\begin{center}
МИНИСТЕРСТВО НАУКИ И ВЫСШЕГО ОБРАЗОВАНИЯ РФ
ФЕДЕРАЛЬНОЕ ГОСУДАРСТВЕННОЕ БЮДЖЕТНОЕ ОБРАЗОВАТЕЛЬНОЕ УЧРЕЖДЕНИЕ
ВЫСШЕГО ОБРАЗОВАНИЯ
"РЯЗАНСКИЙ ГОСУДАРСТВЕННЫЙ РАДИОТЕХНИЧЕСКИЙ УНИВЕРСИТЕТ ИМЕНИ В.Ф. УТКИНА"
\end{center}
\begin{center}
\begin{tabular}{p{0.45\textwidth}p{0.45\textwidth}}
Кафедра: КТ & Дисциплина: Программирование и алгоритмические языки\\
\end{tabular}  
\end{center}  
\begin{center}
БИЛЕТ К ЗАЧЕТУ № \arabic{ticket}
\addtocounter{ticket}{1}

\begin{enumerate}
\item Понятие типа данных. Простые и структурированные типы данных. Языки программирования со слабой и с сильной типизацией данных. Эквивалентность типов данных.
Классификация типов в языке Паскаль.
\item Написать подпрограмму, возвращающее количество элементов одномерного массива V, значение которых превышает среднее арифметическое элементов массива V.
\end{enumerate}
\end{center}
\begin{center}
\begin{tabular}{p{0.45\textwidth}p{0.45\textwidth}}
Зав. кафедрой КТ \_\_\_\_\_\_\_\_\_ & Экзаменатор \_\_\_\_\_\_\_\_\_ \\
Таганов А.И. & доц. каф. КТ Наумов Д.А.\\
\end{tabular}  
\end{center} 

\_\_\_\_\_\_\_\_\_\_\_\_\_\_\_\_\_\_\_\_\_\_\_\_\_\_\_\_\_\_\_\_\_\_\_\_\_\_\_\_\_\_\_\_\_

% Билет 2
\begin{center}
МИНИСТЕРСТВО НАУКИ И ВЫСШЕГО ОБРАЗОВАНИЯ РФ
ФЕДЕРАЛЬНОЕ ГОСУДАРСТВЕННОЕ БЮДЖЕТНОЕ ОБРАЗОВАТЕЛЬНОЕ УЧРЕЖДЕНИЕ
ВЫСШЕГО ОБРАЗОВАНИЯ
"РЯЗАНСКИЙ ГОСУДАРСТВЕННЫЙ РАДИОТЕХНИЧЕСКИЙ УНИВЕРСИТЕТ ИМЕНИ В.Ф. УТКИНА"
\end{center}
\begin{center}
\begin{tabular}{p{0.45\textwidth}p{0.45\textwidth}}
Кафедра: КТ & Дисциплина: Программирование и алгоритмические языки\\
\end{tabular}  
\end{center}  
\begin{center}
БИЛЕТ К ЗАЧЕТУ № \arabic{ticket}
\addtocounter{ticket}{1}

\begin{enumerate}
\item Массив. Синтаксическая форма описания типа данных и переменной-массива.
Доступ к элементам массива. Одномерные и многомерные массивы.
\item Написать подпрограмму вычисления минимального значения произвольной функции F(x), заданной в дискретных точках на интервале (a, b) c шагом h.
\end{enumerate}
\end{center}
\begin{center}
\begin{tabular}{p{0.45\textwidth}p{0.45\textwidth}}
Зав. кафедрой КТ \_\_\_\_\_\_\_\_\_ & Экзаменатор \_\_\_\_\_\_\_\_\_ \\
Таганов А.И. & доц. каф. КТ Наумов Д.А.\\
\end{tabular}  
\end{center} 

\newpage

% Билет 3
\begin{center}
МИНИСТЕРСТВО НАУКИ И ВЫСШЕГО ОБРАЗОВАНИЯ РФ
ФЕДЕРАЛЬНОЕ ГОСУДАРСТВЕННОЕ БЮДЖЕТНОЕ ОБРАЗОВАТЕЛЬНОЕ УЧРЕЖДЕНИЕ
ВЫСШЕГО ОБРАЗОВАНИЯ
"РЯЗАНСКИЙ ГОСУДАРСТВЕННЫЙ РАДИОТЕХНИЧЕСКИЙ УНИВЕРСИТЕТ ИМЕНИ В.Ф. УТКИНА"
\end{center}
\begin{center}
\begin{tabular}{p{0.45\textwidth}p{0.45\textwidth}}
Кафедра: КТ & Дисциплина: Программирование и алгоритмические языки\\
\end{tabular}  
\end{center}  
\begin{center}
БИЛЕТ К ЗАЧЕТУ № \arabic{ticket}
\addtocounter{ticket}{1}

\begin{enumerate}
\item Множественный тип данных. Конструктор множества. 
Синтаксическая форма описания множественного типа и переменной множественного типа.
Операции над множествами в Паскале. Добавление и удаление элементов из множества.
\item В первой строке текстового файла записано число, обозначающее размерность квадратной матрицы. Во второй строке - значения элементов матрицы, разделенные пробелами. 
Определить, содержит ли нули главная диагональ матрицы.
\end{enumerate}
\end{center}
\begin{center}
\begin{tabular}{p{0.45\textwidth}p{0.45\textwidth}}
Зав. кафедрой КТ \_\_\_\_\_\_\_\_\_ & Экзаменатор \_\_\_\_\_\_\_\_\_ \\
Таганов А.И. & доц. каф. КТ Наумов Д.А.\\
\end{tabular}  
\end{center} 

\_\_\_\_\_\_\_\_\_\_\_\_\_\_\_\_\_\_\_\_\_\_\_\_\_\_\_\_\_\_\_\_\_\_\_\_\_\_\_\_\_\_\_\_\_

% Билет 4
\begin{center}
МИНИСТЕРСТВО НАУКИ И ВЫСШЕГО ОБРАЗОВАНИЯ РФ
ФЕДЕРАЛЬНОЕ ГОСУДАРСТВЕННОЕ БЮДЖЕТНОЕ ОБРАЗОВАТЕЛЬНОЕ УЧРЕЖДЕНИЕ
ВЫСШЕГО ОБРАЗОВАНИЯ
"РЯЗАНСКИЙ ГОСУДАРСТВЕННЫЙ РАДИОТЕХНИЧЕСКИЙ УНИВЕРСИТЕТ ИМЕНИ В.Ф. УТКИНА"
\end{center}
\begin{center}
\begin{tabular}{p{0.45\textwidth}p{0.45\textwidth}}
Кафедра: КТ & Дисциплина: Программирование и алгоритмические языки\\
\end{tabular}  
\end{center}  
\begin{center}
БИЛЕТ К ЗАЧЕТУ № \arabic{ticket}
\addtocounter{ticket}{1}

\begin{enumerate}
\item Комбинированный тип данных. Синтаксическая форма описание комбинированного типа данных 
    и переменных комбинированного типа. Оператор with.
\item Написать подпрограмму, формирующую множество гласных букв, входящих в первую строку
   и не входящих во вторую строку.
\end{enumerate}
\end{center}
\begin{center}
\begin{tabular}{p{0.45\textwidth}p{0.45\textwidth}}
Зав. кафедрой КТ \_\_\_\_\_\_\_\_\_ & Экзаменатор \_\_\_\_\_\_\_\_\_ \\
Таганов А.И. & доц. каф. КТ Наумов Д.А.\\
\end{tabular}  
\end{center} 

\newpage

% Билет 5
\begin{center}
МИНИСТЕРСТВО НАУКИ И ВЫСШЕГО ОБРАЗОВАНИЯ РФ
ФЕДЕРАЛЬНОЕ ГОСУДАРСТВЕННОЕ БЮДЖЕТНОЕ ОБРАЗОВАТЕЛЬНОЕ УЧРЕЖДЕНИЕ
ВЫСШЕГО ОБРАЗОВАНИЯ
"РЯЗАНСКИЙ ГОСУДАРСТВЕННЫЙ РАДИОТЕХНИЧЕСКИЙ УНИВЕРСИТЕТ ИМЕНИ В.Ф. УТКИНА"
\end{center}
\begin{center}
\begin{tabular}{p{0.45\textwidth}p{0.45\textwidth}}
Кафедра: КТ & Дисциплина: Программирование и алгоритмические языки\\
\end{tabular}  
\end{center}  
\begin{center}
БИЛЕТ К ЗАЧЕТУ № \arabic{ticket}
\addtocounter{ticket}{1}

\begin{enumerate}
\item Подпрограммы. Назначение. Преимущества использования.
Синтаксическая форма описания функции. Возвращение значения из функции. Вызов функции.
\item В файле целых чисел определить номер первого неотрицательного элемента.
\end{enumerate}
\end{center}
\begin{center}
\begin{tabular}{p{0.45\textwidth}p{0.45\textwidth}}
Зав. кафедрой КТ \_\_\_\_\_\_\_\_\_ & Экзаменатор \_\_\_\_\_\_\_\_\_ \\
Таганов А.И. & доц. каф. КТ Наумов Д.А.\\
\end{tabular}  
\end{center} 

\_\_\_\_\_\_\_\_\_\_\_\_\_\_\_\_\_\_\_\_\_\_\_\_\_\_\_\_\_\_\_\_\_\_\_\_\_\_\_\_\_\_\_\_\_

% Билет 6
\begin{center}
МИНИСТЕРСТВО НАУКИ И ВЫСШЕГО ОБРАЗОВАНИЯ РФ
ФЕДЕРАЛЬНОЕ ГОСУДАРСТВЕННОЕ БЮДЖЕТНОЕ ОБРАЗОВАТЕЛЬНОЕ УЧРЕЖДЕНИЕ
ВЫСШЕГО ОБРАЗОВАНИЯ
"РЯЗАНСКИЙ ГОСУДАРСТВЕННЫЙ РАДИОТЕХНИЧЕСКИЙ УНИВЕРСИТЕТ ИМЕНИ В.Ф. УТКИНА"
\end{center}
\begin{center}
\begin{tabular}{p{0.45\textwidth}p{0.45\textwidth}}
Кафедра: КТ & Дисциплина: Программирование и алгоритмические языки\\
\end{tabular}  
\end{center}  
\begin{center}
БИЛЕТ К ЗАЧЕТУ № \arabic{ticket}
\addtocounter{ticket}{1}

\begin{enumerate}
\item Подпрограммы. Назначение. Преимущества использования. Синтаксическая форма описания процедуры. Возвращение значения из процедуры. Вызов функции.
\item Описать структуру данных для хранения таблицы значений: ФИО сотрудника, год, объем продаж за год. Определить сотрудника с максимальным объемом продаж за 2018 год.
\end{enumerate}
\end{center}
\begin{center}
\begin{tabular}{p{0.45\textwidth}p{0.45\textwidth}}
Зав. кафедрой КТ \_\_\_\_\_\_\_\_\_ & Экзаменатор \_\_\_\_\_\_\_\_\_ \\
Таганов А.И. & доц. каф. КТ Наумов Д.А.\\
\end{tabular}  
\end{center} 

\newpage

% Билет 7
\begin{center}
МИНИСТЕРСТВО НАУКИ И ВЫСШЕГО ОБРАЗОВАНИЯ РФ
ФЕДЕРАЛЬНОЕ ГОСУДАРСТВЕННОЕ БЮДЖЕТНОЕ ОБРАЗОВАТЕЛЬНОЕ УЧРЕЖДЕНИЕ
ВЫСШЕГО ОБРАЗОВАНИЯ
"РЯЗАНСКИЙ ГОСУДАРСТВЕННЫЙ РАДИОТЕХНИЧЕСКИЙ УНИВЕРСИТЕТ ИМЕНИ В.Ф. УТКИНА"
\end{center}
\begin{center}
\begin{tabular}{p{0.45\textwidth}p{0.45\textwidth}}
Кафедра: КТ & Дисциплина: Программирование и алгоритмические языки\\
\end{tabular}  
\end{center}  
\begin{center}
БИЛЕТ К ЗАЧЕТУ № \arabic{ticket}
\addtocounter{ticket}{1}

\begin{enumerate}
\item Обмен данными между программой и подпрограммой. Типы формальных параметров.
Блочный принцип организации программ.
\item Написать подпрограмму сортировки одномерного массива следующим образом: сначала следуют 
 положительные элементы по возрастанию, затем - отрицательные элементы по возрастанию, 
 затем - нулевые элементы.
\end{enumerate}
\end{center}
\begin{center}
\begin{tabular}{p{0.45\textwidth}p{0.45\textwidth}}
Зав. кафедрой КТ \_\_\_\_\_\_\_\_\_ & Экзаменатор \_\_\_\_\_\_\_\_\_ \\
Таганов А.И. & доц. каф. КТ Наумов Д.А.\\
\end{tabular}  
\end{center} 

\_\_\_\_\_\_\_\_\_\_\_\_\_\_\_\_\_\_\_\_\_\_\_\_\_\_\_\_\_\_\_\_\_\_\_\_\_\_\_\_\_\_\_\_\_

% Билет 8
\begin{center}
МИНИСТЕРСТВО НАУКИ И ВЫСШЕГО ОБРАЗОВАНИЯ РФ
ФЕДЕРАЛЬНОЕ ГОСУДАРСТВЕННОЕ БЮДЖЕТНОЕ ОБРАЗОВАТЕЛЬНОЕ УЧРЕЖДЕНИЕ
ВЫСШЕГО ОБРАЗОВАНИЯ
"РЯЗАНСКИЙ ГОСУДАРСТВЕННЫЙ РАДИОТЕХНИЧЕСКИЙ УНИВЕРСИТЕТ ИМЕНИ В.Ф. УТКИНА"
\end{center}
\begin{center}
\begin{tabular}{p{0.45\textwidth}p{0.45\textwidth}}
Кафедра: КТ & Дисциплина: Программирование и алгоритмические языки\\
\end{tabular}  
\end{center}  
\begin{center}
БИЛЕТ К ЗАЧЕТУ № \arabic{ticket}
\addtocounter{ticket}{1}

\begin{enumerate}
\item Процедурный тип данных. Передача подпрограммы как параметра.
\item Написать подпрограмму, возвращающую минимальный отрицательный элемент массива. Если 
в массиве нет отрицательных элементов, функция должна возвращать 0.
\end{enumerate}
\end{center}
\begin{center}
\begin{tabular}{p{0.45\textwidth}p{0.45\textwidth}}
Зав. кафедрой КТ \_\_\_\_\_\_\_\_\_ & Экзаменатор \_\_\_\_\_\_\_\_\_ \\
Таганов А.И. & доц. каф. КТ Наумов Д.А.\\
\end{tabular}  
\end{center} 

\newpage

% Билет 9
\begin{center}
МИНИСТЕРСТВО НАУКИ И ВЫСШЕГО ОБРАЗОВАНИЯ РФ
ФЕДЕРАЛЬНОЕ ГОСУДАРСТВЕННОЕ БЮДЖЕТНОЕ ОБРАЗОВАТЕЛЬНОЕ УЧРЕЖДЕНИЕ
ВЫСШЕГО ОБРАЗОВАНИЯ
"РЯЗАНСКИЙ ГОСУДАРСТВЕННЫЙ РАДИОТЕХНИЧЕСКИЙ УНИВЕРСИТЕТ ИМЕНИ В.Ф. УТКИНА"
\end{center}
\begin{center}
\begin{tabular}{p{0.45\textwidth}p{0.45\textwidth}}
Кафедра: КТ & Дисциплина: Программирование и алгоритмические языки\\
\end{tabular}  
\end{center}  
\begin{center}
БИЛЕТ К ЗАЧЕТУ № \arabic{ticket}
\addtocounter{ticket}{1}

\begin{enumerate}
\item Модульное программирование. Понятие модуля. Структура модуля в языке Паскаль.
\item Написать подпрограмму определения номера столбца с максимальной суммой элементов.
\end{enumerate}
\end{center}
\begin{center}
\begin{tabular}{p{0.45\textwidth}p{0.45\textwidth}}
Зав. кафедрой КТ \_\_\_\_\_\_\_\_\_ & Экзаменатор \_\_\_\_\_\_\_\_\_ \\
Таганов А.И. & доц. каф. КТ Наумов Д.А.\\
\end{tabular}  
\end{center} 

\_\_\_\_\_\_\_\_\_\_\_\_\_\_\_\_\_\_\_\_\_\_\_\_\_\_\_\_\_\_\_\_\_\_\_\_\_\_\_\_\_\_\_\_\_

% Билет 10
\begin{center}
МИНИСТЕРСТВО НАУКИ И ВЫСШЕГО ОБРАЗОВАНИЯ РФ
ФЕДЕРАЛЬНОЕ ГОСУДАРСТВЕННОЕ БЮДЖЕТНОЕ ОБРАЗОВАТЕЛЬНОЕ УЧРЕЖДЕНИЕ
ВЫСШЕГО ОБРАЗОВАНИЯ
"РЯЗАНСКИЙ ГОСУДАРСТВЕННЫЙ РАДИОТЕХНИЧЕСКИЙ УНИВЕРСИТЕТ ИМЕНИ В.Ф. УТКИНА"
\end{center}
\begin{center}
\begin{tabular}{p{0.45\textwidth}p{0.45\textwidth}}
Кафедра: КТ & Дисциплина: Программирование и алгоритмические языки\\
\end{tabular}  
\end{center}  
\begin{center}
БИЛЕТ К ЗАЧЕТУ № \arabic{ticket}
\addtocounter{ticket}{1}

\begin{enumerate}
\item Файл. Файловый тип данных. Основные подпрограммы работы с файлами.
\item Билет на автобус содержит шесть цифр. Минимальный номер билета - 000001, максимальный - 999999. Билет счастливый, если сумма первой тройки цифр равна сумма второй тройке цифр.
Определить количество счастливых билетов.
\end{enumerate}
\end{center}
\begin{center}
\begin{tabular}{p{0.45\textwidth}p{0.45\textwidth}}
Зав. кафедрой КТ \_\_\_\_\_\_\_\_\_ & Экзаменатор \_\_\_\_\_\_\_\_\_ \\
Таганов А.И. & доц. каф. КТ Наумов Д.А.\\
\end{tabular}  
\end{center} 

\newpage

% Билет 11
\begin{center}
МИНИСТЕРСТВО НАУКИ И ВЫСШЕГО ОБРАЗОВАНИЯ РФ
ФЕДЕРАЛЬНОЕ ГОСУДАРСТВЕННОЕ БЮДЖЕТНОЕ ОБРАЗОВАТЕЛЬНОЕ УЧРЕЖДЕНИЕ
ВЫСШЕГО ОБРАЗОВАНИЯ
"РЯЗАНСКИЙ ГОСУДАРСТВЕННЫЙ РАДИОТЕХНИЧЕСКИЙ УНИВЕРСИТЕТ ИМЕНИ В.Ф. УТКИНА"
\end{center}
\begin{center}
\begin{tabular}{p{0.45\textwidth}p{0.45\textwidth}}
Кафедра: КТ & Дисциплина: Программирование и алгоритмические языки\\
\end{tabular}  
\end{center}  
\begin{center}
БИЛЕТ К ЗАЧЕТУ № \arabic{ticket}
\addtocounter{ticket}{1}

\begin{enumerate}
\item Режимы открытия файлов. Закрытие файлов. Операции ввода-вывода с файлами.
Контроль ошибок при работе с файлами.
\item Написать подпрограмму определения номера строки с минимальным количество отрицательных элементов.
\end{enumerate}
\end{center}
\begin{center}
\begin{tabular}{p{0.45\textwidth}p{0.45\textwidth}}
Зав. кафедрой КТ \_\_\_\_\_\_\_\_\_ & Экзаменатор \_\_\_\_\_\_\_\_\_ \\
Таганов А.И. & доц. каф. КТ Наумов Д.А.\\
\end{tabular}  
\end{center} 

\_\_\_\_\_\_\_\_\_\_\_\_\_\_\_\_\_\_\_\_\_\_\_\_\_\_\_\_\_\_\_\_\_\_\_\_\_\_\_\_\_\_\_\_\_

% Билет 12
\begin{center}
МИНИСТЕРСТВО НАУКИ И ВЫСШЕГО ОБРАЗОВАНИЯ РФ
ФЕДЕРАЛЬНОЕ ГОСУДАРСТВЕННОЕ БЮДЖЕТНОЕ ОБРАЗОВАТЕЛЬНОЕ УЧРЕЖДЕНИЕ
ВЫСШЕГО ОБРАЗОВАНИЯ
"РЯЗАНСКИЙ ГОСУДАРСТВЕННЫЙ РАДИОТЕХНИЧЕСКИЙ УНИВЕРСИТЕТ ИМЕНИ В.Ф. УТКИНА"
\end{center}
\begin{center}
\begin{tabular}{p{0.45\textwidth}p{0.45\textwidth}}
Кафедра: КТ & Дисциплина: Программирование и алгоритмические языки\\
\end{tabular}  
\end{center}  
\begin{center}
БИЛЕТ К ЗАЧЕТУ № \arabic{ticket}
\addtocounter{ticket}{1}

\begin{enumerate}
\item Статическое и динамическое выделение памяти. Указатели. Типизированные и нетипизированные указатели.
\item Написать подпрограмму, определяющую сумму элементов, лежащих выше главной диагонали 
квадратной матрицы.
\end{enumerate}
\end{center}
\begin{center}
\begin{tabular}{p{0.45\textwidth}p{0.45\textwidth}}
Зав. кафедрой КТ \_\_\_\_\_\_\_\_\_ & Экзаменатор \_\_\_\_\_\_\_\_\_ \\
Таганов А.И. & доц. каф. КТ Наумов Д.А.\\
\end{tabular}  
\end{center} 

\end{document}